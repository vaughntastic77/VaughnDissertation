%%%%%%%%%%%%%%%%%%%%%%%%%%%%%%%%%%%%%%%%%%%%%%%%%%%
%
%  Author: Jacob Vaughn
%  
%  Last Updated: 3/8/2024
%
%%%%%%%%%%%%%%%%%%%%%%%%%%%%%%%%%%%%%%%%%%%%%%%%%%%
%%%%%%%%%%%%%%%%%%%%%%%%%%%%%%%%%%%%%%%%%%%%%%%%%%%%%%%%%%%%%%%%%%%%%%
%%                         CONCLUSIONS
%%%%%%%%%%%%%%%%%%%%%%%%%%%%%%%%%%%%%%%%%%%%%%%%%%%%%%%%%%%%%%%%%%%%%%

\chapter{REAMAINING WORK}

The path forward will follow the manufacturing schedule shown in Figure \ref{fig:schedule}. The pressure test will be performed as soon as the nozzles are finished machining, tentatively May 1, 2024. Depending on the results of the pressure test, the process could be anywhere from a single day to two weeks. Once complete though, the nozzles and sidewalls will be shipped to the polishing vendor, which should have a quick turn around of two weeks. Upon arrival to the NAHL, the full ACE2.0 nozzle will be finally assembled and installed. With proper planning prior to the final installation, the process should not take more than a week to begin shakedown and characterization.

In the meantime, the servo control program will be written between now and the pressure test. The Mach number feedback will be implemented, and the Reynolds number control capability will be explored and potentially implemented before the final install. Again, this will primarily depend on the test schedule for the M6QT and the amount of time needed to replace the manual valve with a controlled valve. 

The characterization and hysteresis experiments will be performed immediately following installation and initial calibration. The primary goal is to ensure complete function of ACE2.0 along with a demonstration of the capabilities and potential for future research. Additionally, the documentation for the nozzle operation will be written based on the best practices deduced throughout the experiments.

%%%%%%%%%%%%%%%
% End of body %
%%%%%%%%%%%%%%%

\nocite{aiaa-uncertainty-standard}
\nocite{anderson-fundamentals}
\nocite{anderson-compressible}
