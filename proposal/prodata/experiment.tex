%%%%%%%%%%%%%%%%%%%%%%%%%%%%%%%%%%%%%%%%%%%%%%%%%%%
%
%  Author: Jacob Vaughn
%  
%  Last Updated: 1/13/2024
%
%%%%%%%%%%%%%%%%%%%%%%%%%%%%%%%%%%%%%%%%%%%%%%%%%%%
%%%%%%%%%%%%%%%%%%%%%%%%%%%%%%%%%%%%%%%%%%%%%%%%%%%%%%%%%%%%%%%%%%%%%%
%%                       EXPERIMENTAL SETUP
%%%%%%%%%%%%%%%%%%%%%%%%%%%%%%%%%%%%%%%%%%%%%%%%%%%%%%%%%%%%%%%%%%%%%

\chapter{EXPERIMENTAL APPROACH \& OBJECTIVES}

\section{Nozzle Noise and Uniformity Characterization}

In order to establish a baseline for future work within the ACE2.0 facility, a pitot survey was performed to measure and characterize the freestream noise and uniformity throuhgout the nozzle. Utilize pitot probe/rake and kulites mounted on traverse to characterize entire nozzle exit plane and centerline into nozzle up to 24??? inches upstream of nozzle exit.

\subsection{Noise Hysterisis}

Sweep Re/m and Mach to explore hysteresis of noise and maybe uniformity. Begin with Re/m sweep up and back down in current ACE.

\section{Model Heating Hysteresis}

Look into hysteresis of heating, boundary layers, and shocks using IR imaging. 

\section{Experimental Control and Efficiency Improvements} 

Stuff about ACE2.0 controls

\subsection{Feedback Controlled Mach Selection}

Inputting Mach number into PLC to actively adjust to desired Mach number within some tolerance. Result of above ACE2.0 development.

\subsection{Re/m Control Scheme}

Control $P_{0}$ to maintain constant Re/m during Mach sweep. Look into anticapatory change in Re/m for potential delayed response time in $P_{0}$, and look into potential proportional Re/m change to model acceleration or altitude change.

