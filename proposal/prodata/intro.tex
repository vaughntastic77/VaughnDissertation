%%%%%%%%%%%%%%%%%%%%%%%%%%%%%%%%%%%%%%%%%%%%%%%%%%%
%
%  Author: Jacob Vaughn
%  
%  Last Updated: 3/8/2024
%
%%%%%%%%%%%%%%%%%%%%%%%%%%%%%%%%%%%%%%%%%%%%%%%%%%%

%%%%%%%%%%%%%%%%%%%%%%%%%%%%%%%%%%%%%%%%%%%%%%%%%%%%%%%%%%%%%%%%%%%%%%
%%                         INTRODUCTION
%%%%%%%%%%%%%%%%%%%%%%%%%%%%%%%%%%%%%%%%%%%%%%%%%%%%%%%%%%%%%%%%%%%%%


\pagestyle{plain} % No headers, just page numbers
\pagenumbering{arabic} % Arabic numerals
\setcounter{page}{1}


\chapter{INTRODUCTION \& LITERATURE REVIEW}

\section{Introduction}

In recent decades, the continual improvement in hypersonic aircraft aerodynamics has emphasized the need for advancements in wind tunnel ground testing capabilities \cite{leyva}. The conventional approach, reliant on distinct nozzles for discrete Mach numbers, poses logistical challenges and limits the exploration of dynamic characteristics in evolving aircraft designs. Recognizing these limitations, there is a growing demand within the hypersonic community for a novel solution, namely a continuously variable Mach-number nozzle designed to seamlessly adapt to the evolving needs of hypersonic research.

This imperative shift towards innovation seeks to overcome the constraints of conventional wind tunnels by introducing a continuously variable Mach-number nozzle to provide researchers with a more similar representation of real-world scenarios. By dynamically adjusting the Mach number throughout the wind tunnel runs, the variable conditions experienced by hypersonic vehicles during different flight trajectories can be replicated. This capability enables the advancement of ground testing for a more comprehensive understanding of dynamic hypersonic phenomena.

The Actively Controlled Expansion (ACE) wind tunnel at Texas A\&M University has served as a workhorse in hypersonic research for over a decade, but it is overdue for improvements to meet the growing demand of hypersonic flight research. Although the facility was initially designed to facilitate the continuous variation of Mach number, the mechanical implementation ultimately proved to be overly simplistic. Consequently, the nozzle predominantly maintained a fixed Mach 6 setting throughout the majority of the tunnel's operation, falling short of fully realizing its designated variable nature.

\section{Research Outline and Objectives}

In order to maintain the National Aerothermochemistry and Hypersonics Laboratory (NAHL) as a cutting-edge research facility, its current facilities are rapidly advancing in capability to enable better science. The objectives of this research aim to lay the foundation for varaible Mach number wind tunnel control to meet the recent increased demand of dynamic hypersonic vehicle aerodynamics research.

The existing ACE facility will be upgraded to achieve true active control and to potentially produce low-distrubance flow for higher Reynolds numbers. Its successor, ACE2.0, will employ a feedback-control system with servo motors, linear actuators, and various instrumentation to enable the accurate and continuous variation of Mach nuber and Reynolds number. Once frabricated and calibrated, the ACE2.0 facility will be utilized to accomplish the following objectives:

\begin{enumerate}
    \item Improved experimental control and efficiency
        \begin{enumerate}
            \item Feedback-controlled active Mach number selection
            \item Constant/proportional Reynolds number control
        \end{enumerate}
    \item Characterization of noise and uniformity throughout nozzle
        \begin{enumerate}
            \item Uncerntainty quantification
            \item Hysteresis investigation
        \end{enumerate}
    \item Preliminary investigation of model flow characteristics hysteresis during Mach trajectory and oscillation
\end{enumerate}

These objectives will effectively demonstrate the capabilities and merit of the new ACE2.0 facility. The intent is to evaluate the performance of the facility while paving the groundwork for the next decade of dynamic hypersonic flight research. In addition, the standard operating procedures for ACE2.0 will be updated to reflect the best practices deduced throughout the completion of these objectives, and the resulting control procedures and interface will be straightforward and well documented for future student researches to easily learn and utilize, ensuring a seamless transition for future investigations. The documentation will not only enhance the accessibility of ACE2.0 for subsequent research endeavors but also contribute to the broader scientific community by providing a robust framework for effective wind tunnel control and dynamic hypersonic vehicle aerodynamics exploration.

\section{Literature Review}

The literature review for this dissertation will be examined in four parts related to hypersonic variable Mach-number wind tunnels and according to the above objectives: (1) variable mach number nozzle design, (2) parameter control, (3) flow characterization and uncerntainty, and (4) hysteresis in hypersonic flows. This review will discuss articles that establish the most current knowledge base and techniques in the relevant areas of hypersonic wind tunnel research.

Variable Mach number nozzles have been explored in many configurations since the 1950s such as interchangeable fixed-block, plug-type, asymmetric sliding blocks, tilting plate, fully flexible, and hinged/flexure \cite{agard-ag-3}. Each of these designs have varying degrees of flow quality, cost effectiveness, and experimental efficiency that must be considered. Only the fully flexible and flexure designs maximize experimental efficiency without sacrificing flow quality. Of these two, the flexure design minimizes costs by reducing mechanical complexity and supporting structure. Therefore, the flexure design is the optimal choice considering these criteria.

The flexure type nozzle was first proposed in 1955 by Rosen \cite{rosen} and improved upon separately by Erdmann and Rom \cite{erdmann,rom} in order to minimize the mechanical complexity. This simple nozzle design operated by a single jack greatly reduces manufactring and controls costs and allows for greater flexibilty in active control to quickly and continuously vary the Mach number to model dynamic supersonic vehicle flight.

In the last decade, many variable mach number supersonic wind tunnels have been manufactured due to increased demand of hypersonic flight research. The majority of these are fully flexible or flexure nozzle designs with varying implementations of actuation and control \cite{durand,laguarda,chen,guo,lv,qi,steeves}. All of these facilites were developed to study vehicle flight trajectory and the hysteresis phenomenon therein.

With the increased emergence of these variable mach number facilites, effective control schemes must be employed for the flow paramters $M$, $P_0$, $T_0$, and the resulting $Re/m$ in order to vary each parameter independently and accurately model hypersonic flight conditions through various trajectories by maintaining flow similarity. This control problem, acknowledged as early as the 1980s, prompted the development of diverse solutions implementing the various areas of control theory such as optimal control \cite{kraft,hwang}, state feedback control, mathematical model prediction control, preprogrammed controllers \cite{matsumoto}, and PID control \cite{fung,ilic-2}.

In recent years, researchers at numerous state-of-the-art variable Mach number facilities have embraced advanced intelligent control methods. Techniques such as fuzzy logic, genetic algorithms, neural networks, adaptive control or gain scheduling, and their combinations have been applied \cite{nott,shahrbabaki-1}, reflecting a contemporary shift towards leveraging intelligent algorithms to address the complexities and nonlinearity of hypersonic wind tunnel flow control. The methods that will be explored in this research are those of Hwang \cite{hwang}, Matsumoto \cite{matsumoto}, Ili\'c \cite{ilic-2}, and Shahrbabaki \cite{shahrbabaki-1} as they each introduce the different advantages and challenges of each control technique. 

First, Hwang developed a robust LQG/LTR based controller enhanced by an anti-integrator windup and a modified Smith predictor to overcome unavoidable modeling errors, uncerntainties, and time-delay effects. This controller demonstrated a faster stabilization and exhibited fewer oscillations in comparison to its PID counterpart. Given its superior performance, it presents an appealing prospect for implementation in ACE2.0, and a detailed exploration of this controller will be undertaken in a subsequent chapter.

Next, Matsumoto took a simplified approach by replacing an existing real-time PID controller with a preprogrammed controller to avoid input time delays. This was advantagous for his facility becuase the run time was not much longer than the time delay for the PID controller to stabilize. This is the most straightforward approach to obtain specific constant or dynamic trajectories of multiple input parameters, but it is not without its challenges. The controller must have a new program for each individual desired parameter set condition or path, and each program must be iterated to minimize errors. Additionally, considering the longer run times of ACE2.0, a PID controller has ample time to stabillize and can be implemeted. 

Then, Ili\'c implemented a cascase nonlinear feedforward-feedback PID controller as a combined system to enhance a standard single-loop PID. The systems setpoint reference tracking is improved by the feedforward-feedback architecture, and the distrubance rejection is improved by the cascade architecture. With these two architectures combined in one multi-loop controller, large transient overshoots are eliminated, setpoint settling times are decreased, and the overall accuracy of the controlled parameters is maximized. Once again, the improved performance of this controller makes it another appealing prospect for ACE2.0, which will be discussed later.

Lastly, Shahrbabaki utilized an artificial neural network and fuzzy logic to enhance a conventional PD controller to handle the complex nonlinearity of the variable mach number wind tunnel flow parameters. The advantages of fuzzy logic include its simplicity and adaptabliity of introducing new control rules to handle imprecise data, uncerntainty, and unmodeled dynamics. The combined advantage that Shahrbabaki explores pertains to the utilization of the neural network to develop the membership functions for the fuzzy logic controller. He designed and trained a feed-forward multilayer perceptron neural network according to the database from the mathematical model of the wind tunnel behavior in order to develop the optimal membership functions. This method will only be explored further for ACE2.0 if the methods of Hwang or Ili\'c do not yield sufficient performance.

Additionally, with parameter control introduced in a hypersonic wind tunnel, the uncerntainty of the various flow parameters can be quantified more effectively. The primary references for the uncerntainty quantification in this research will be the NASA report by Stephens \cite{stephens-hubbard} and Hubbard, Chair of AIAA Wind Tunnel Measurement Uncerntainty Committee on Standards, and the dissertation by Curriston in 2024 \cite{curriston}. The methodology in this report combines the techinques of the prevelant literature on the subject from the last few decades, and Curriston demonstrates this methodolgy in essentially a case study for reference.

Now, considering flow characterization in literature, the primary references will be the recent AIAA articles by Chou \cite{chou} and Duan \cite{duan} on hypersonic wind tunnel freestream disturbance measurements as they clearly provide the latest measurement processes and procedures and reference over 50 publications on relevant topics from the last couple decades. In addition to these two references, a decade of NAHL experience and best practices will guide the characterization of ACE2.0 upon its fabrication and initial shakedown.

Finally, the review of hypersonic flow hysteresis in literature yielded many publications discussing the phenomenon primarily in shock interactions and inlet start/unstart processes. The inlet literature will not be referenced directly in this work, but it will undoubtedly be invaluable for future research in ACE2.0. Focusing on the shock interactiions, both numerical and experimental data is presented throughout this literature. Hysteresis has been reported in hypersonic wind tunnel experiments as early as the 1950s \cite{kenworthy,beastall}. The test conditions that produced hysteresis were usually avoided in experiments until the 1990s when the phenomona began to be studied directly \cite{chpoun,ben-dor-1,durand}. Recent literature reveals numerical investigations easily reproduced shock interaction hysteresis, while experimental investigations proved more difficult to reproduce the hysteresis due to the freestream noise in convential facilities \cite{laguarda}. Nevertheless, hysteresis was succesfully observed experimentally in low-noise (quiet) wind tunnels \cite{ivanov,setoguchi,tao}. Methodolgies from all of this lierature will be studied in order to attempt to reproduce shock interaction hysteresis in ACE2.0. Additionally, the data gathered by Wirth \cite{wirth} in the existing ACE facility will serve as the primary reference for the exploration of surface heat flux hysteresis of a fin-cone model. 

Something about modern design of experiemtns (MDOE) here with reference to DeLoach.
