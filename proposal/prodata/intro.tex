%%%%%%%%%%%%%%%%%%%%%%%%%%%%%%%%%%%%%%%%%%%%%%%%%%%
%
%  Author: Jacob Vaughn
%  
%  Last Updated: 3/8/2024
%
%%%%%%%%%%%%%%%%%%%%%%%%%%%%%%%%%%%%%%%%%%%%%%%%%%%

%%%%%%%%%%%%%%%%%%%%%%%%%%%%%%%%%%%%%%%%%%%%%%%%%%%%%%%%%%%%%%%%%%%%%%
%%                         INTRODUCTION
%%%%%%%%%%%%%%%%%%%%%%%%%%%%%%%%%%%%%%%%%%%%%%%%%%%%%%%%%%%%%%%%%%%%%


\pagestyle{plain} % No headers, just page numbers
\pagenumbering{arabic} % Arabic numerals
\setcounter{page}{1}


\chapter{INTRODUCTION \& LITERATURE REVIEW}

\section{Introduction}

In recent decades, the continual improvement in hypersonic aircraft aerodynamics has emphasized the need for advancements in wind tunnel ground testing capabilities \cite{leyva}. Conventional hypersonic wind tunnels rely on distinct fixed nozzle contours to uniformly accelerate the flow to the desired Mach number. This approach fixes the Mach number and only provides a particular flow regime for experiments. Recognizing this, there is a clear need for a continuously variable Mach-number nozzle designed to overcome the limitations of conventional wind tunnels and enable more advanced dynamic hypersonic research.

The objective of this work is to introduce a continuously variable and actively controllable Mach-number nozzle. By dynamically adjusting the Mach number throughout the wind tunnel runs, the variable conditions experienced by hypersonic vehicles during different flight trajectories can be replicated. This capability enables the advancement of ground testing for a more comprehensive understanding of dynamic hypersonic flight and associated phenomena. Furthermore, the active control capability will increase experimental efficiency by allowing measurements at different Mach numbers within a single run and introduce the ability to fine tune the Mach number for improved data quality.

The Actively Controlled Expansion (ACE) wind tunnel at Texas A\&M University has served as a workhorse in hypersonic research for over a decade \cite{ace09,ace10-calibrate,tichenor-dis,mai-dis,neel-dis,leidy-dis}, but it is overdue for improvements to meet the growing demand of hypersonic flight research. Although the facility was initially designed to facilitate the continuous variation of Mach number, the mechanical implementation ultimately proved to be cumbersome to adjust. Consequently, the nozzle has remained fixed at Mach 6 for the majority of the tunnel's operation, falling short of fully realizing its designated variable Mach capability. Additionally, despite being fixed for a constant Mach number, the value varies throughout a run by up to 5\%. Considering this, it is apparent that an update to the ACE nozzle is necessary to remedy these shortcomings.

\section{Research Objectives}

The objectives of this research aim to lay the foundation for continuously variable Mach number wind tunnel control at the National Aerothermochemistry and Hypersonics Laboratory (NAHL). Doing so will expand the current capabilities within the lab for more advanced hypersonic flight experiments. This will help maintain the NAHL as a cutting-edge national research facility.

The existing ACE facility will be upgraded to achieve true active control and to potentially produce low-disturbance flow for higher Reynolds numbers. Its successor, ACE2.0, that is the subject of this work, will employ a feedback-control system with servo motors, linear actuators, and various instrumentation to enable the accurate and continuous variation of Mach number and Reynolds number. Once fabricated and calibrated, the ACE2.0 facility will provide:

\begin{enumerate}[listparindent=\parindent]
    \item Improved experimental control and efficiency
            
        The control system mentioned above will be implemented to enable feedback-controlled active Mach number variation and selection for accurate Mach trajectories and set points during a run. The feedback aspect will attempt to control the Mach number to within 0.1\% of the set value.
        
        Similarly, a control scheme will also be developed to enable feedback-controlled Reynolds number variation and selection that responds to changes in Mach number and stagnation temperature for accurate sweeps and set points. This will allow both a constant or proportional Reynolds number during a Mach trajectory to maintain similarity.

        The control of both of these parameters will yield improved experimental efficiency with a new capability to explore multiple flow configurations within a single run. Besides enhancing efficiency, the Mach number and Reynolds number control will enable more robust uncertainty quantification and more dynamic experiments that were not possible before. Both of these capabilities are demonstrated in the next objectives.

    \item Characterization of freestream pressure fluctuation levels and flow uniformity throughout the nozzle and uncertainty quantification of flow parameters

        A pitot survey of the nozzle exit plane and centerline will be conducted to measure the freestream pressure fluctuation levels and uniformity throughout the nozzle and characterize its performance. This will validate the design and manufacturing of the nozzle and settling chamber and provide a basis for the quality of data gathered in future experiments.

        Taking one step further, a rigorous uncertainty analysis will be performed to quantify the systematic and random uncertainty of the measured flow parameters $P$, $P_0$, and $T_0$ and the resulting values of Mach number, $M$, and unit Reynolds number, $Re'$. This will establish the baseline uncertainty for the freestream flow parameters and enable improved data quality for future experiments.

        In order to fully characterize the tunnel behavior while actively controlled, an investigation of the potential existence of hysteresis phenomena will be performed. If discovered, any hysteresis will be characterized to fully understand the dynamics of the freestream flow as each parameter is varied.

    \item Demonstration of Mach trajectory and potential hysteresis phenomenon in proof of concept experiment

        The capabilities of ACE2.0 will be demonstrated in an experiment that will showcase shock wave interactions between two wedges during a Mach trajectory. This experiment was chosen to explore the well-known hysteresis in the transition from a regular reflection to a mach reflection and the ability to reproduce the phenomenon in this facility. 

\end{enumerate}

These objectives will effectively validate and demonstrate the capabilities and merit of the new ACE2.0 facility. In addition, the standard operating procedures for ACE2.0 will be updated to reflect the best practices deduced throughout the completion of these objectives. The resulting control procedures and interface will be straightforward and well documented for future student researches to easily learn and utilize, ensuring a seamless transition for future investigations. The documentation will not only enhance the accessibility of ACE2.0 for subsequent research endeavors but also contribute to the broader scientific community by providing a robust framework for effective wind tunnel control and dynamic hypersonic vehicle aerodynamics exploration.

\section{Literature Review}

The literature review for this dissertation includes four parts related to hypersonic variable Mach-number wind tunnels and according to the above objectives: (1) variable mach number nozzle design, (2) parameter control, (3) flow quality characterization and uncertainty quantification, and (4) hysteresis in hypersonic flows. This review will discuss articles that establish the most current knowledge base and techniques in the relevant areas of hypersonic wind tunnel research.

\subsubsection{Variable Mach-Number Wind Tunnels}
Variable Mach number nozzles have been explored in many configurations since the 1950s such as interchangeable fixed-block, plug-type, asymmetric sliding blocks, tilting plate, fully flexible, and hinged/flexure \cite{agard-ag-3}. Each of these designs have varying degrees of flow quality, cost effectiveness, and experimental efficiency that must be considered. Only the fully flexible and flexure designs maximize experimental efficiency without sacrificing flow quality. Of these two, the flexure design minimizes costs by reducing mechanical complexity and supporting structure. Therefore, the flexure design is the optimal choice considering these criteria.

The flexure type nozzle was first proposed in 1955 by Rosen \cite{rosen} and improved upon separately by Erdmann in 1971\cite{erdmann} and Rom and Etsion in 1972 \cite{erdmann,rom} in order to minimize the mechanical complexity. This simple nozzle design operated by a single jack greatly reduces manufacturing and controls costs and allows for greater flexibility in active control to quickly and continuously vary the Mach number to model dynamic supersonic vehicle flight.

In the last decade, many variable mach number supersonic wind tunnels have been manufactured due to increased demand of hypersonic flight research. The majority of these are fully flexible or flexure nozzle designs with varying implementations of actuation and control \cite{ilic-1,shahrbabaki-1,durand,laguarda,chen,guo,lv,qi,steeves}. All of these facilities were developed to study vehicle flight trajectory and the hysteresis phenomenon therein.

The ACE tunnel, the facility of interest for this work, is a flexure type nozzle that was designed and manufactured between 2009 and 2010 and began operating in 2010 \cite{ace09,ace10-calibrate,tichenor-dis}. 

\textcolor{red}{Include paragraph or two about ACE with plots???}

\subsubsection{Parameter Control}
With the increased emergence of these variable mach number facilities, effective control schemes must be employed for the controllable parameters $A^*$, $P_0$, $T_0$, and the resulting Mach number, M, and unit Reynolds number, $Re'$, in order to vary each parameter independently and accurately model hypersonic flight conditions through various trajectories by maintaining flow similarity. This control problem, acknowledged as early as the 1980s, prompted the development of diverse solutions implementing the various areas of control theory such as optimal control \cite{kraft,hwang}, state feedback control, mathematical model prediction control, preprogrammed controllers \cite{matsumoto}, and PID control \cite{fung,ilic-2,silva}.

In recent years, researchers at numerous state-of-the-art variable Mach number facilities have embraced advanced intelligent control methods. Techniques such as fuzzy logic, genetic algorithms, neural networks, adaptive control or gain scheduling, and their combinations have been applied \cite{nott,shahrbabaki-1}, reflecting a contemporary shift towards leveraging intelligent algorithms to address the complexities and nonlinearity of hypersonic wind tunnel flow control. The methods that will be explored in this research are those of Hwang et al. \cite{hwang}, Matsumoto et al. \cite{matsumoto}, Ili\'c et al. \cite{ilic-2}, and Shahrbabaki et al. \cite{shahrbabaki-1} as they each introduce the different advantages and challenges of each control technique. 

First, Hwang et al. developed a robust LQG/LTR based controller enhanced by an anti-integrator windup and a modified Smith predictor to overcome unavoidable modeling errors, uncertainties, and time-delay effects. This controller demonstrated a faster stabilization and exhibited fewer oscillations in comparison to its PID counterpart. Given its superior performance, it presents an appealing prospect for implementation in ACE2.0, and a detailed exploration of this controller will be undertaken in a subsequent chapter.

Matsumoto et al. took a simplified approach by replacing an existing real-time PID controller with a preprogrammed controller to avoid input time delays. This was advantageous for his facility because the run time was not much longer than the time delay for the PID controller to stabilize. This is the most straightforward approach to obtain specific constant or dynamic trajectories of multiple input parameters, but it is not without its challenges. The controller must have a new program for each individual desired parameter set condition or path, and each program must be iterated to minimize errors. Additionally, considering the longer run times of ACE2.0, a PID controller has ample time to stabilize and can be implemented. 

Ili\'c et al. implemented a cascade nonlinear feedforward-feedback PID controller as a combined system to enhance a standard single-loop PID. The systems set point reference tracking is improved by the feedforward-feedback architecture, and the disturbance rejection is improved by the cascade architecture. With these two architectures combined in one multi-loop controller, large transient overshoots are eliminated, set point settling times are decreased, and the overall accuracy of the controlled parameters is maximized. Once again, the improved performance of this controller makes it another appealing prospect for ACE2.0, which will be discussed later.

Shahrbabaki et al. utilized an artificial neural network and fuzzy logic to enhance a conventional PD controller to handle the complex nonlinearity of the variable mach number wind tunnel flow parameters. The advantages of fuzzy logic include its simplicity and adaptability of introducing new control rules to handle imprecise data, uncertainty, and unmodeled dynamics. The combined advantage that Shahrbabaki et al. explores pertains to the utilization of the neural network to develop the membership functions for the fuzzy logic controller. He designed and trained a feed-forward multilayer perceptron neural network according to the database from the mathematical model of the wind tunnel behavior in order to develop the optimal membership functions. This method will only be explored further for ACE2.0 if the methods of Hwang et al. or Ili\'c et al. do not yield sufficient performance.

\subsubsection{Flow Quality Characterization and Uncertainty Quantification}
The primary references regarding flow quality characterization will be the recent AIAA articles by Chou et al. \cite{chou} and Duan et al. \cite{duan} on hypersonic wind tunnel freestream disturbance measurements. These provide the latest measurement processes and procedures and reference over 50 publications on relevant topics from recent decades. In addition to these two references, a decade of NAHL experience and best practices will guide the characterization of ACE2.0 upon its fabrication and initial shakedown. \textbf{Key NAHL ACE references are...}

The primary references for the uncertainty quantification in this research will be the NASA report by Stephens et al. \cite{stephens-hubbard} with coauthor Hubbard, Chair of AIAA Wind Tunnel Measurement Uncertainty Committee on Standards, and the dissertation by Curriston \cite{curriston}. The methodology in this NASA report combines the techniques of the prevalent literature on the subject from the last few decades to quantify the uncertainty of the flow parameters in a supersonic wind tunnel. Curriston's work provides a secondary reference as he thoroughly demonstrates this methodology as a case study in a low speed wind tunnel. Additionally, Leidy \cite{leidy-dis} performed a very conservative uncertainty analysis in the existing ACE tunnel that will provide a rough baseline reference for the uncertainty quantification for ACE2.0. 

\subsubsection{Hysteresis in Hypersonic Flows}
Finally, the review of hypersonic flow hysteresis in literature yielded many publications discussing the phenomenon primarily in shock interactions and inlet start/unstart processes. The inlet start/unstart literature will not be referenced directly in this work, but it will undoubtedly be invaluable for future research in ACE2.0. Focusing on the shock interactions, both numerical and experimental data is presented throughout the literature. The first report of shock wave reflection phenomenon was by Ernst Mach in 1878, and the various aspects and configurations were studied extensively by Von Neumann in 1943 \cite{von-neumann}. Following Von Neumann's work, Hornung et al. predicted hysteresis in the transition from regular reflection to Mach reflection in 1979 \cite{hornung-1}, but he was unable to experimentally produce the hysteresis effect throughout his years of experiments \cite{hornung-2}. Recent literature reveals numerical investigations easily reproduce shock interaction hysteresis \cite{chpoun-1,ivanov-3}, while experimental investigations prove more difficult to reproduce the hysteresis due to the freestream noise in conventional facilities \cite{ben-dor-1,laguarda}. Nevertheless, hysteresis has been successfully observed and studied experimentally in open and low-noise (quiet) wind tunnels since 1995 \cite{chpoun-2}. The two processes that produce hysteresis in the shock interactions are wedge-angle variation and Mach-number variation \cite{ben-dor-2}. Since it is significantly more complex to vary the Mach number, most experimental results are due to the wedge-angle-variation-induced hysteresis \cite{li,ivanov-4,mouton,setoguchi,chanetz}. However, some research groups with variable Mach number tunnels were able to produce shock interaction hysteresis experimentally by varying the Mach number \cite{durand,tao}. Methodologies from the numerical and experimental literature on Mach-number-variation-induced hysteresis will be studied in order to attempt to reproduce shock interaction hysteresis in ACE2.0.

\textcolor{red}{Something about modern design of experiments (MDOE) with reference to DeLoach.}
