%%%%%%%%%%%%%%%%%%%%%%%%%%%%%%%%%%%%%%%%%%%%%%%%%%%
%
%  Author: Jacob Vaughn
%  
%  Last Updated: 1/13/2024
%
%%%%%%%%%%%%%%%%%%%%%%%%%%%%%%%%%%%%%%%%%%%%%%%%%%%

%%%%%%%%%%%%%%%%%%%%%%%%%%%%%%%%%%%%%%%%%%%%%%%%%%%%%%%%%%%%%%%%%%%%%%
%%                           INTRODUCTION
%%%%%%%%%%%%%%%%%%%%%%%%%%%%%%%%%%%%%%%%%%%%%%%%%%%%%%%%%%%%%%%%%%%%%


\pagestyle{plain} % No headers, just page numbers
\pagenumbering{arabic} % Arabic numerals
\setcounter{page}{1}


\chapter{INTRODUCTION}

\section{Hypersonics}

The exploration of hypersonic flow regimes has become increasingly critical in advancing aerospace technologies, where vehicles operate at speeds ranging from Mach 5 to 8. Hypersonic wind tunnels serve as indispensable tools for studying these extreme conditions, facilitating the development of aerodynamic designs and materials capable of withstanding the unique challenges posed by such velocities. This dissertation delves into the design, fabrication, and characterization of a Mach 5 to 8 wind tunnel, with a primary focus on enhancing experimental control and efficiency.

\section{Hypersonic Wind Tunnels}

The realm of hypersonics introduces an array of complexities that demand meticulous investigation. Traditional wind tunnels fall short in accurately replicating the dynamic conditions experienced at these velocities. This dissertation addresses this gap by introducing a novel wind tunnel design that incorporates an actively controlled Mach number through a sophisticated servo control system. This innovation not only offers a platform for more precise experimentation but also lays the groundwork for advancements in hypersonic flight technology.

\section{Turbulence}

Understanding turbulence and transition phenomena is paramount in comprehending the aerodynamic behaviors of objects traversing hypersonic environments. The second section of this introduction explores the intricate interplay between Mach numbers, unit Reynolds numbers, and the dynamic nature of turbulence. A comprehensive survey, including both static and dynamic assessments, will unravel the nuances of noise and uniformity during a Mach number sweep, providing essential insights for the subsequent hysteresis study.

\section{Research Objectives}

The following objectives stuff:

\begin{enumerate}
    \item Improve experimental control and efficiency
    \item Feedback controlled Mach selection
    \item Characterization of noise and uniformity throughout nozzle with hysteresis
    \item Model BL, shock, and heating hysteresis
    \item Constant/proportional Re/m
\end{enumerate}
